\section{Лекция 19.04.19}

\textit{Продолжение свойств УМО}

\begin{enumerate}
    \setcounter{enumi}{8}
    \item Предельный переход под знаком под знаком МО.
        \begin{enumerate}
            \item Пусть \(\xi_n > 0\), \(\xi_n \downarrow \xi\) монотонно, \(\E{\xi} < \infty\), тогда \(\E{\xi_n | \eta} \uparrow \E{\xi | \eta}\).
            \item Если \(\xi_n \to \xi\) п.н., \(\abs{\xi_n}, \abs{\xi} < \delta,\) \(\E{\delta} < \infty\), то
            \begin{displaymath}
                \E{\xi_n | \eta} \overset{\text{п. н.}}\to \E{\xi | \eta}.
            \end{displaymath}
        \end{enumerate}
        \begin{enumerate}
            \item \begin{proof}
                В силу свойства 5 (про сохранение порядка) \(\E{\xi_n | \eta} < \E{\xi_{n + 1} | \eta}\). Тогда можно рассмотреть \(\zeta = \lim_{n \to \infty} \E{\xi_n | \eta}\). Проверим, что \(\zeta = \E{\xi | \eta}\). Заметим, что \(\E{\xi_n | \eta}\) --- борелевские функции от \(\eta\), откуда \(\zeta \in B(\R)\) от \(\eta\). Проверим интегральное свойство:
                \begin{multline*}
                    \forall B \in B(\R) \xi_n \I{\eta \in B} \uparrow \xi \I{\eta \in B} \implies \\ \implies
                    \set{\text{по теореме о монотонной сходимости}} \implies \\ \implies
                    \E{\xi \I{\eta \in B}} = \lim_{n \to \infty} \E{\xi_n \I{\eta \in B}} = \\ =
                    \set{\text{интегральное свойство}} = \\ =
                    \lim_{n} \E{\E{\xi_n | \eta} \I{\eta \in B}} = \\ =
                    \abs{\E{\xi_n | \eta) \I{\eta \in B} \uparrow \zeta \I{\eta \in B}}} = \\ =
                    \set{\text{теорема о монотоннй сходимости}} = \lim_n \E{\zeta \I{\eta \in B}}.
                \end{multline*}
            \end{proof}
            \item \begin{proof}
                Рассмотрим \(\zeta_n = \sup_{m \leq n} \abs{\xi_m - \xi}\). Тогда \(\abs{\zeta_n} \leq 2\delta\) и \(\zeta_n \downarrow 0\) с вероятностью 1.
                Тогда по пункту (a), \(\E{\zeta_n | \eta} \downarrow 0\) почти наверное.
                Но \(\abs{\E{\xi_n | \eta} - \E{\xi | \eta}}\) в силу свойства 6 не больше \(\E{\abs{\xi_n - \xi} | \eta}\), что не больше в силу 5 \(\E{\zeta_n | \eta} \to 0\).
            \end{proof}
        \end{enumerate}

\item Если \(\delta = f(\eta)\) --- борелевская функция, \(\E{\abs{\xi \delta}} < \infty\), \(\E{\abs{\xi}} < \infty\), то
    \begin{displaymath}
        \E{\xi \delta | \eta} = \delta \E{\xi | \eta}.
    \end{displaymath}
    \begin{proof}
        Пусть сначала \(\delta = f(\eta)\), \(f(x) = \I{x \in A}\) --- простая функция, \(A \in B(\R)\). Тогда: правая часть есть функция от \(\eta\), осталось проверить интегральной свво.
        \begin{multline*}
            \E{\xi \delta \I{\eta \in B}} = \E{\xi \I{\eta \in A} \I{\eta \in B}} = \\ = \E{\xi \I{\eta \in A \cap B}} = \set{\text{интегральное свво}} = \\ = \E{\E{\xi | \eta} \I{\eta \in A \cap B}} = \E{\E{\xi | \eta} \delta \I{\eta \in B}}.
        \end{multline*}

        В силу линейности, равенство \(\E{\xi \delta | \eta} = \delta \E{\xi | \eta}\) останется верным для простых функций \(f(x) = \sum_{k = 1}^N c_k \I{x \in B_k}\), где \(c_k \in \R, B_k = B(\R)\).
        
        Если \(f(x)\) произвольная, то возьмем последовательность простых \(f_n(x)\): \(f_n(x) \to f(x)\) и  \(\abs{f_n(x)} \leq \abs{f(x)}\). По свойству 8,
        \begin{displaymath}
            \E{\xi \delta | \eta} = \lim_n \E{\xi f_n(\eta) | \eta} = \E{\xi | \eta} \lim_n f_n(\eta) = \delta \E{\xi | \eta}.
        \end{displaymath}
    \end{proof}

\item (Неравенство Йенсена). Если \(f(x)\) -- выпуклая вниз, то
    \begin{displaymath}
        \E{f(\xi) | \eta} \geq f(\E{\xi | \eta}).
    \end{displaymath}

    \begin{proof}
        Аналогично нормальному нер. Йенсена

        Воспользуемся свойством 9. (Рисунок параболы; в любой точке функция лежит выше, чем прямая: \(\forall x \in \R \exists \lambda(x): \forall y \in \R f(y) \geq f(x) + \lambda(x)(y - x)\)). Положим \(y = \xi\), \(x = \E{\xi | \eta}\). Берем УМО от обеих частей:
        \begin{displaymath}
            \E{f(\xi) | \eta} \geq f(\E{\xi | \eta}) + \E{\lambda \E{\xi | \eta} (\xi - \E{\xi | \eta}) | \eta}.
        \end{displaymath}
        Последнее равно 0: \(\lambda \E{\xi | \eta}\) --- борелевская от \(\eta\); по свойству 9, последнее слагаемое равно:
        \begin{displaymath}
            \lambda \E{\xi | \eta} (\E{\xi | \eta} - \E{\xi | \eta}) = 0.
        \end{displaymath}
    \end{proof}

    \item Идея: знаем \(Y\), хотим получить прогноз на \(X\) функцией \(f(Y)\). Минимизируем \(\E{X - f(Y)}^2 \to \min_f\). Ответ: УМО.
        \begin{theorem}[О наилучшем квадратичном прогнозе]
            Пусть у \(X\) конечный второй момент: \(\E{X^2} < \infty\). Тогда
            \begin{displaymath}
                \E{X - \E{X | Y}}^2 = \min_{f \in \set{\text{борелевская}}} \E{X - f(Y)}^2,
            \end{displaymath}
            причем минимум единственен с точностью до равенства п.н.
        \end{theorem}

        \begin{proof}
            Пусть \(f\) произвольная борелевская.
            \begin{multline*}
                \E{X - f(Y)}^2 = \E{X - \E{X | Y} + \E{X | Y} - f(Y)}^2 = \\
                = \E{X - \E{X | Y}}^2 + \E{\E{X | Y} - f(Y)}^2 + 2 \E{(X - \E{X | Y})(\E{X | Y} - f(Y))}.
            \end{multline*}
            Первое слагаемое хотим, второе больше 0, третье равно 0: воспользуемся формулой полной вероятности.
            \begin{displaymath}
                \E{(X - \E{X | Y})(\E{X | Y - f(Y)})} = \set{\text{фпв}} = \E{\E{(X - \E{X | Y}) (\E{X | Y} - f(Y)) | Y}}
            \end{displaymath}
            Посмотрим на \(\E{X | Y} - f(Y)\) --- функция от \(Y\). По свойству 9, выносим за УМО:
            \begin{displaymath}
                \E{(\E{X | Y} - f(Y)) \E{(X - \E{X | Y}) | Y}} = \set{\text{по линейности}} = \E{X | Y} - \E{X | Y} = 0.
            \end{displaymath}
            Тем самым, минимум достигается тогда и только тогда \(f(Y) = \E{X | Y}\) с вероятностью 1.
        \end{proof}

        \begin{remark}
            Эта теорема будет фундаментом построения байесовских оценок.
        \end{remark}
\end{enumerate}

\subsection{Условное распределение}

Как считать УМО? Если \(Y\) дискретная св, то
\begin{displaymath}
    \E{X | Y} = \sum_{y} \frac{\E{x \I{Y = y}}}{\Pr{Y = y}} \I{Y = y}.
\end{displaymath}
Хотим обобщить на непрерывные \(Y\). Естественно обобщить:
    \begin{displaymath}
        \E{X | Y = y} \overset{def}= \frac{\E{X \I{Y = y}}}{\Pr{Y = y}} = \phi(y).
    \end{displaymath}
Заметим, что \(\E{X | Y} = \phi(Y) \iff \E{X | Y = y} = \phi(y)\).
\begin{definition}[Условное математическое ожидаение \(\E{X | Y = y}\)]
    \(\E{X | Y = y}\) --- борелевская функция \(\phi(y)\) такая, что \(\forall B \in B(\R)\)
    \begin{displaymath}
        \E{X \I{Y \in B}} = \E{\phi(X) \I{Y \in B}}.
    \end{displaymath}
\end{definition}

\begin{remark}
    \(\E{X | Y} = \phi(Y) \iff \E{X | Y = y} = \phi(y)\)
\end{remark}

\begin{definition}[Условное распределение]
    Условным распределением \(X\) при условии \(Y = y\) называется
    \begin{displaymath}
        \Pr{X \in B | Y = y} \overset{def}= \E{\I{X \in B} | Y = y},
    \end{displaymath}
    рассматриваемая как функция от \(B \in B(\R)\).
\end{definition}

\begin{enumerate}
    \item Для \(\forall B \in B(\R)\) \(P{X \in B | Y = y}\) есть борелевская функция от \(y\).
    \item Для \(\forall y \in R\) \(\Pr{x \in B | Y = y}\) есть вероятностная мера на \((\R, B(\R))\).
    \item Если \(Y\) имеет плотность \(\p{Y}{y}\), то
        \begin{displaymath}
            \Pr{X \in B, Y \in A} = \int_{A} \Pr{x \in B | Y = y} \p{Y}{y} \diff y.
        \end{displaymath}
\end{enumerate}

Пункты 1 и 3 понятны, 2 не будем доказывать.

\begin{theorem}[Условное распределение существует]
    Условное распределение существует.
\end{theorem}

\begin{definition}[Условная плотность]
    Если условное распределение \(\Pr{X \in B | Y = y}\) имеет плотность, то есть
    \begin{displaymath}
        \Pr{X \in B | Y = y} = \int_{B} \p{X | Y}{x | y} \diff x,
    \end{displaymath}
    то \(\p{X | Y}(x | y)\) называется условной плотностью \(X\) относительно \(Y\).
\end{definition}

\begin{remark}
    Условная плотность нужна для вычисления УМО, как и обычная плотность для обычного МО.
\end{remark}

\begin{theorem}[О вычислении УМО]
    Если существует условная плотность \(\p{X | Y}{x | y}\), то для любой борелевской функции \(g(x)\) выполнено:
    \begin{displaymath}
        \E{g(X) | Y = y} = \int_\R g(x) \p{X | Y}{x | y} \diff x.
    \end{displaymath}
\end{theorem}

\begin{proof}
    Для УМО \(\E{\xi | \eta = y}\) выполнены свойства обычного УМО: линейность, сохранение отношения порядка, теорема о предельном переходе и так далее.

    Пусть сначала \(g(x)\) индикатор \(g(x) = \I{x \in A}\), \(A \in B(\R)\). Тогда:
    \begin{multline*}
        \E{g(X) | Y = y} = \E{\I{X \in A} | Y = y} = \Pr{X \in A | Y = y} = \\ = \set{\text{определение условной плотности}} = \int_A \p{X | Y}{x | y} \diff x = \\ = \int_\R \I{x \in A} \p{X | Y}{x | y} \diff x = \int_\R g(x) \p{X | Y}{x | y} \diff x.
    \end{multline*}

    Если \(g(x)\) простая, то все верно в силу линейности обеих частей по \(g\).

    Если \(g(x)\) произвольная, то приближаем ее простыми и пользуемся свойством о предельном переходе.
\end{proof}

\subsection{Нахождение условной плотности}
Условная плотность --- непрерывный аналог формулы условной вероятности.

\begin{theorem}[Достаточное условие существования условной плотности]
    Пусть \(X, Y\) случайные величины, имеют совместную плотность \(\p{X, Y}{x, y}\). Тогда функция 
    \begin{displaymath}
    \p{X | Y}{x | y} = \begin{cases}
        \frac{\p{X, Y}{x, y}}{\p{Y}y}, &\text{if } \p{Y}{y} > 0 \\
        0 &\text{otherwise}.
    \end{cases}
    \end{displaymath}
\end{theorem}
\begin{proof}
    Проверим, что
    \begin{displaymath}
        \int_B \p{X | Y}{x | y} \diff x
    \end{displaymath}
    есть условное распределение \(X\) при условии \(Y = y\). Проверяем три свойства: борелевская функция от \(Y\), вероятностная мера --- интеграл равен 1. Проверим свойство 3:
    \begin{multline*}
        \Pr{X \in B, Y \in A} = \int_{B \times A} \p{X, Y}{x, y} \diff x \diff y = \\ = \set{\text{пофубиним!}} = \int_A \left(\int_B \p{X, Y}{x, y} \diff x\right) \diff y = \\ = \int_A \left(\int_B \p{X | Y}{x | y} \diff x\right) \p{Y}y \diff y.
    \end{multline*}
\end{proof}

