\section{Лекция 6}

\subsection{Условное математическое ожидание}

Пусть $X, Y$ -- дискретные случайные простые(с конечным числом значений) величины. Тогда матожидание определяем как:
$$ \E X = \sum_{i = 1}^{n} x_iP(X = x_i)
$$
Вопрос: как в данном случае определить матожидание $X$ при условии, что $Y = y$?
Ответ:
\begin{definition}
\emph{Условным математическим ожиданием} называется величина:
$$
\E(X|Y = y) = \sum_{i = 1}^{n} x_iP(X = x_i|Y = y)
$$
\end{definition}
Вопрос: как определять, если X -- не дискретная случайная величина(Y -- дискретная)? Преобразуем определение для двух дискретных велчин, чтобы понять, как можно определить УМО в данном случае:
$$
\E(X|Y = y) = \sum_{i = 1}^{n} x_iP(X = x_i|Y = y) = \sum_{i = 1}^{n} \frac{x_iP(X = x_i, Y = y)}{P(Y = y)} = \sum_{i = 1}^{n} \frac{x_i\E I\{X = x_i, Y = y\}}{P(Y = y)} = $$
$$= \frac{\E (\sum_{i = 1}^{n} x_i I\{X = x_i, Y = y\})}{P(Y = y)} = \frac{E(XI\{Y = y\})}{P(Y = y)}
$$
Теперь пусть $Y$ тоже не дискретная, тогда преобразуем получившееся сверху определение, чтобы понять, как можно определить УМО в данном случае:
Сперва заметим, что:
$$ \frac{E(XI\{Y = y\})}{P(Y = y)} = f(y)
$$
Тогда
$$ \sum_{y \in B} f(y)P(Y = y) = \E (XI\{Y \in B\})
$$
С другой стороны: 
$$ \sum_{y \in B} f(y)P(Y = y) = \E(f(y)I\{Y \in B\})
$$
Значит, 
$$
\E (XI\{Y \in B\}) = \E(f(y)I\{Y \in B\})
$$
Теперь мы готовы дать определение условного математического ожидания в общем случае.
\begin{definition}
Пусть X, Y -- случайные величины. \emph{Условным математическим ожиданием $X$ относительно $Y$} называется случайная величина $\E(X|Y)$, обладающая двумя свойствами:
\begin{itemize}
    \item Измеримость. То есть $\E(X|Y)$ борелевская функция от $Y$.
    \item Интегральное свойство. А именно
    $$
    \E (XI\{Y \in B\}) = \E(\E(X|Y)I\{Y \in B\})
    $$
    \end{itemize}
\end{definition}

\begin{remark}
Если $Y$ -- дискретная случайная величина, то
$$
\E(X|Y) = \sum_{i = 1}^{m} \frac{\E XI\{Y = y_i\}}{P(Y = y_i)}I\{Y = y_i\}
$$
\end{remark}

\begin{remark}
Важно понимать, что $\E(X|Y)$ -- это именно \emph{случайная величина}:
$$
\E(X|Y = y) = f(y) \Leftrightarrow \E(X|Y) = f(Y)
$$
\end{remark}

Возникает вопрос о корректности определения условного математического ожидания. На этот вопрос есть ответ без доказательства.
\begin{theorem}[О существовании условного математического ожидания]
Если $\E X < \infty$, то для любой случайной величины $Y$ условное матожидание $\E(X|Y)$ существует и единственно(с точностью до равенства п.н.).
\end{theorem}

\begin{remark}[Смысл УМО]
УМО -- усреднение $X$ по значениям $Y$.
\end{remark}

\begin{example}
Вот здесь на лекции была картинка.
\end{example}

\subsection{Свойства УМО}
\begin{enumerate}
    \item Если $X = \phi(Y)$, где $\phi$ -- борелевская функция, то
    $$
    \E(X|Y) = X = \phi(Y)
    $$
    \begin{remark}[Смысл]
    Ожидание функции от $Y$ при условии $Y$, равно самой функции. 
    \end{remark}
    \item $\E(\E(X|Y)) = \E X$
    \begin{proof}
    Воспользуемся интегральным свойством при $B = \R$. Так как $I\{Y \in \R\} \equiv 1$, то
    $$
    \E (XI\{Y \in B\}) = \E (X) = \E(\E(X|Y)I\{Y \in B\}) = \E(\E(X|Y))
    $$
    \end{proof}
    \item Линейность. То есть
    $$
    \E(aX + bZ|Y) = a\E(X|Y) + b\E(Z|Y) 
    $$
    \begin{proof}
    Убедимся, что выражение справа удовлетворяет свойствам \\УМО(измеримости и интегральному свойсву).\\
    Заметим, что $a\E(X|Y) + b\E(Z|Y)$ удовлетворяет свойству измеримости. Проверим интегральное свойство:\\
    $\forall B \in \mathcal{B}(\R)$
    $$
    \E((aX + bZ)I\{Y \in B\}) = a\E(XI\{Y \in B\}) + b\E(ZI\{Y \in B\}) = $$
    $$ =  a\E(\E(X|Y)I\{Y \in B\}) + b\E(\E(Z|Y)I\{Y \in B\}) = \E((a\E(X|Y) + b\E(Z|Y))I\{Y \in B\})$$
    \end{proof}
    \item Если $X$ и $Y$ независимы, то $\E(X|Y) = \E X$. 
    \begin{proof}
    $\E X$ -- константа, а значит и борелевская функция от $Y$. Проверим интегральное свойство:
    $\forall B \in \mathcal{B}(\R)$
    $$
    \E(XI\{Y \in B\}) = \E(X)\E I\{Y \in B\}= \E(\E X I\{Y \in B\})
    $$
    \end{proof}
    \item Если $X \leq Z$ почти наверное, то $\E(X|Y) \leq \E(Z|Y)$ почти наверное.
    \begin{proof}
    $\forall B \in \mathcal{B}(\R)$
    $$ \E(XI\{Y \in B\}) \leq \E(ZI\{Y \in B\})\Rightarrow
    $$
    В силу интегрального свойства:
    $$
    \E(\E(X|Y)I\{Y \in B\}) \leq \E(\E(Z|Y)I\{Y \in B\}) \Rightarrow \E(\E(Z|Y)I\{Y \in B\}) - \E(\E(X|Y)I\{Y \in B\}) \geq 0 $$
    Пусть случайная величина $\Psi(Y) = \E(Z|Y) - \E(X|Y)$, тогда
    $$
    \E(\Psi(Y) I\{Y \in B\}) = \E(\E(Z|Y)I\{Y \in B\}) - \E(\E(X|Y)I\{Y \in B\}) \geq 0
    $$
    значит, по свойству матожидания $\Psi(Y) \geq 0$ почти наверное.
    \end{proof} 
    \item $|\E(X|Y)| \leq \E(|X|$ $|Y)$
    \begin{proof}
    Следует из свойств 5 и 3, так как $-|X| \leq X \leq |X|$
    \end{proof}
    \begin{remark}
    Если Y -- случайный вектор, то $\E(X|Y)$ определяется точно также, достаточно $\R$ заменить на $\R^n$.
    \end{remark}
    \item Телескопическое свойство:
    $$1) \E(\E(X|Y)|Y, Z) = \E(X|Y)
    $$
    $$2) \E(\E(X|Y, Z)|Y) = \E(X|Y)
    $$
    \begin{remark}
    Y, Z -- это вектор из случайных величин Y и Z.
    \end{remark}
    \begin{proof}
    1) Поскольку $\E(X|Y)$ -- борелевская функция от $Y$, то она также является борелевской функцией от вектора $(X, Y)$. Тогда по 1 свойству УМО:
    $$
    \E(\E(X|Y)|Y, Z) = \E(X|Y)
    $$
    2) Проверим, что $\E(X, Y)$ <<подходит>> на роль УМО. $\E(X, Y)$ -- это борелевская функция от Y. Проверим интегральное свойство: \\
    $\forall B \in \mathcal{B}(\R)$
    $$ \E((X|Y, Z)I\{(Y, Z) \in B \times \R \} = |\text{интегральное свойство}| = \E(XI\{(Y, Z) \in B \times \R \}) = $$ 
    $$ = \E(XI\{Y \in B\}) = |\text{интегралное свойство}| = \E(\E(X|Y)I\{Y \in B\} 
    $$
    \end{proof}
\end{enumerate}